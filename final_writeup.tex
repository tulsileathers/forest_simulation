\documentclass{article}

\usepackage{fancyhdr}
\usepackage{extramarks}
\usepackage{amsmath}
\usepackage{amsthm}
\usepackage{amsfonts}
\usepackage{tikz}
\usepackage[plain]{algorithm}
\usepackage{algpseudocode}

\usetikzlibrary{automata,positioning}

%
% Basic Document Settings
%

\topmargin=-0.45in
\evensidemargin=0in
\oddsidemargin=0in
\textwidth=6.5in
\textheight=9.0in
\headsep=0.25in

\linespread{1.1}

\pagestyle{fancy}
\lhead{\hmwkAuthorName\ \hmwkClass\ (\hmwkClassInstructor\ \hmwkClassTime): \hmwkTitle}
\rhead{\firstxmark}
\lfoot{\lastxmark}
\cfoot{\thepage}

\renewcommand\headrulewidth{0.4pt}
\renewcommand\footrulewidth{0.4pt}

\setlength\parindent{0pt}


%
% Homework Details
%   - Title
%   - Due date
%   - Class
%   - Section/Time
%   - Instructor
%   - Author
%

\newcommand{\hmwkTitle}{Final Project}
\newcommand{\hmwkDueDate}{Dec 16, 2015}
\newcommand{\hmwkClass}{Modeling, Simulation, and Analysis}
\newcommand{\hmwkClassTime}{Section 1}
\newcommand{\hmwkClassInstructor}{Professor Eric Aaron}
\newcommand{\hmwkAuthorName}{Jayce Rudig-Leathers}

%
% Title Page
%

\title{
    \vspace{2in}
    \textmd{\textbf{\hmwkClass:\ \hmwkTitle}}\\
    \normalsize\vspace{0.1in}\small{Due\ on\ \hmwkDueDate\ at 10:30am}\\
    \vspace{0.1in}\large{\textit{\hmwkClassInstructor\ \hmwkClassTime}}
    \vspace{3in}
}

\author{\textbf{\hmwkAuthorName}}
\date{}



\begin{document}

\maketitle

\pagebreak
\vspace{.25cm}
\section*{Proposal}
I propose to construct a cellular automata based model of a forest to answer
questions about forest dynamics, including competition among different species for
resources such as light and space. I believe that cellular automata model will allow
 my project to have the further scope in the time I have available to it. Some examples
 of what I'd like to find out with my model are: how a forest regrows after clear cutting,
 which type of trees are more apt to grow back. What effect does the max height/width of a tree
have on its competitive ability. I will use Lett et al.'s  model as a basis for constructing
my own.






\end{document}
