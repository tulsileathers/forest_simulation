\documentclass{article}
\usepackage{cite}
\usepackage{fancyhdr}
\usepackage{extramarks}
\usepackage{amsmath}
\usepackage{amsthm}
\usepackage{amsfonts}
\usepackage{tikz}
\usepackage[plain]{algorithm}
\usepackage{algpseudocode}
\usepackage{array}

\usetikzlibrary{automata,positioning}

%
% Basic Document Settings
%

\topmargin=-0.45in
\evensidemargin=0in
\oddsidemargin=0in
\textwidth=6.5in
\textheight=9.0in
\headsep=0.25in

\linespread{1.1}

\pagestyle{fancy}
\lhead{\hmwkAuthorName\ \hmwkClass\ (\hmwkClassInstructor\ \hmwkClassTime): \hmwkTitle}
\rhead{\firstxmark}
\lfoot{\lastxmark}
\cfoot{\thepage}

\renewcommand\headrulewidth{0.4pt}
\renewcommand\footrulewidth{0.4pt}

\setlength\parindent{0pt}


%
% Homework Details
%   - Title
%   - Due date
%   - Class
%   - Section/Time
%   - Instructor
%   - Author
%

\newcommand{\hmwkTitle}{Final Project}
\newcommand{\hmwkDueDate}{Dec 16, 2015}
\newcommand{\hmwkClass}{Modeling, Simulation, and Analysis}
\newcommand{\hmwkClassTime}{Section 1}
\newcommand{\hmwkClassInstructor}{Professor Eric Aaron}
\newcommand{\hmwkAuthorName}{Jayce Rudig-Leathers}

%
% Title Page
%

\title{
    \vspace{2in}
    \textmd{\textbf{\hmwkClass:\ \hmwkTitle}}\\
    \normalsize\vspace{0.1in}\small{Due\ on\ \hmwkDueDate\ at 10:30am}\\
    \vspace{0.1in}\large{\textit{\hmwkClassInstructor\ \hmwkClassTime}}
    \vspace{3in}
}

\author{\textbf{\hmwkAuthorName}}
\date{}



\begin{document}

\maketitle

\pagebreak
\vspace{.25cm}
\section*{Proposal}
I propose to construct a cellular automata based model of a forest to answer
questions about forest dynamics, including competition among different species for
resources such as light and space. I believe that a cellular automata model will allow
 my project to have the furthest scope in the time I have available to it. Some examples
 of what I'd like to find out with my model are: how a forest regrows after clear cutting,
 which type of trees are more apt to grow back. What effect does the max height/width of a tree
have on its competitive ability.

\section*{The model}
I am using Lett's  model as a basis for my own \cite{Lett1999277}. Presented here are the
equations and constants I used in my simulation. In the interest of time
I will only state the differences between my model and Lett's. Lett's model does
not contain values for the last seven values in the Constants table. These I intuited
or found in the case of k which I found as online as the average yearly light extinction for all biomes.


\section*{Species Specific parameters}
\begin{center}
  \begin{tabular}{|l|p{5cm}|c|c|r|}
    \hline
     & Description &White birch &	Yellow birch &	Beech\\
    G &	Parameter that determines how early in its age (or size) a tree achieves most of its growth &	190.1 &	143.6 &	87.7\\
    Dmax &	Maximum diameter at breast height (cm) &	76 &	100 &	160\\
    Hmax &	Maximum height (cm) &	3050 &	3050 &	3660\\
    AGEmax & Maximum age (years) &	140	 & 300 &	366\\
    C &	Constant of proportionality relating leaf area to tree diameter &	0.486	 & 0.486  &	2.200\\
    Almin &	 Minimum proportion of incident sunlight needed for regeneration &	0.99  &	0.90 &	0.00\\
    Almax &	 &	1.00 &	0.99 &	0.90\\
    b2 &	 &	76.6 &	58.3 &	44.0\\
    b3 &	 &	0.504 &	0.291 &	0.137\\
    \hline
  \end{tabular}
\end{center}

\section*{Equations}
\begin{center}
  \begin{tabular}{|>{$}c<{$}|r|}
    \hline
    \text{Equation} & Description \\
    \hline
    \Delta D_{opt} = \frac{GD \left (1 - \frac{DH}{D_{max}H_{max}} \right )}{274 +3b_{2}D - 4b_{3}D^{2}} & Optimal Growth for a tree \\
    \hline
    AL = AL_{0}exp(- kSLAI) & Available Light \\
    \hline
    BAR = \sum_{neighborhood} \frac{\pi D^{2}}{4} & Total Basal Area Equation \\
    \hline
    f(AL)_{1-2} = 2.24(1 - exp(-1.136(AL - .08))) & Shade response for shade intolerant and shade intermediat species \\
    \hline
    f(AL)_{3} = 1 - exp( - 4.64(AL - .05)) & Shade response for shade tolerant species \\
    \hline
    s(BAR)  = 1 - \frac{BAR}{BAR_{max}} & Response to space competition\\
    \hline
    \Delta D = \Delta D_{opt} * f(AL) * s(BAR) & Change in diameter \\
    \hline
    D(t + 1) = D(t) + \Delta D & Equation for Diameter \\
    \hline
    H(t + 1) = 137 + b_{2}D(t + 1) - b_{3}D(t + 1)^{2} & Equation for height \\
    \hline
    LA(t + 1) = CD(t + 1)^{2} & Equation for leaf area \\
    \hline
  \end{tabular}
\end{center}

\section*{Constants}
\begin{center}
  \begin{tabular}{|>{$}c<{$}|>{$}c<{$}|p{5cm}|}
    \hline
    \text{Parameter} & \text{Value} & Descripton \\
    \hline
    dt & 1 & timestep in years \\
    \hline
    simulationLength & 200 & simulation length \\
    \hline
    numIterations  &  simulationLength/dt & number of iterations \\
    \hline
    m &  25 & size of forest in cells\\
    \hline
    n  &  25 & size of forest in cells \\
    \hline
    R  &  6 & neighborhood radius meters \\
    \hline
    s_{cell}  &  2 & size of cell in meters \\
    \hline
    AL0 & .99 & light incident above canopy \\
    \hline
    k & .56 & light extinction coefficient  \\
    \hline
    BARmax & 150 & maximum basal area  \\
    \hline
    ageDeathFraction  &  .001 & fraction of trees to reach species dependent maximum age \\
    \hline
    competitionDeathFraction  &  .01 & fraction of trees which should die if growth is below AINC \\
    \hline
    AINC  &  .001 & minimum D increment to not have a chance of death \\
    \hline
    competitiveDeathProbability  &  1 - competitionDeathFraction^(1/10) & chance of competitive death \\
    \hline
  \end{tabular}
\end{center}
\section*{Validation of Model}
I'm currently in the process of model validation. My main method of validation is to
get results in line with Lett's. Currently I'm experiencing unpredictable results from
my simulation. There's an error somewhere and I'm trying to find it. I believe it's to do
with my choice of constants and how I'm utilizing them.

\section*{Evaluation of Model}
It's too early to truly evaluate the model but some initial thoughts I've had on
the structure of Lett's model as I've been working are as follows. \\

1. The other two trees within a cell are not taken into account within the neighbor
equation. It seems to me that for any tree in a cell, the other two trees in that
cell should be its neighbors as well as those within the radius. 

  \bibliographystyle{plain}
  \bibliography{science}
\end{document}
